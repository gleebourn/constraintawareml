\documentclass[10pt,a4paper]{article}
\pdfoutput=1
\usepackage[margin=1in,footskip=0.25in]{geometry}

\usepackage{amsmath}
\usepackage{amssymb}
\usepackage{amsthm}
\usepackage{mathtools}
\usepackage{xparse}
\usepackage{cite}
\usepackage{hyperref}
\usepackage{cleveref}
\usepackage{geometry}
\usepackage{pgfplots}
\usepackage{listings}
\usepackage{tikz}
\usepackage{stmaryrd}
\usepackage{float}
\usetikzlibrary{positioning}
\usetikzlibrary{calc}
\usetikzlibrary{quotes}
%\usepackage{sansmathfonts}
\usepackage[T1]{fontenc}

\pgfplotsset{compat=1.16}

\newtheorem{claim}{Claim}
\newtheorem{rmk}{Remark}
\newtheorem{defn}[claim]{Definition}
\newtheorem{eg}[claim]{Example}
\newtheorem{lem}[claim]{Lemma}
\newtheorem{thm}[claim]{Theorem}
\newtheorem{cor}[claim]{Corollary}

\DeclareMathOperator{\spn}{span}
\DeclareMathOperator{\diag}{diag}
\DeclareMathOperator{\prj}{Proj}
\DeclareMathOperator{\mat}{Mat}
\DeclareMathOperator{\codim}{codim}

\DeclareMathOperator{\be}{\text{\fontencoding{X2}\selectfont б}}
\DeclareMathOperator{\ve}{\text{\fontencoding{X2}\selectfont в}}
\DeclareMathOperator{\ghe}{\text{\fontencoding{X2}\selectfont г}}

\newcommand{\xrightarrowdbl}[2][]{
  \xrightarrow[#1]{#2}\mathrel{\mkern-14mu}\rightarrow
}

\NewDocumentCommand \bern { O{k} }{
    \rho_{#1}
}

\NewDocumentCommand \calLp { O{n} O{\omega} }{
    {\mathcal L'}^{(#1)}_{#2}
}
\NewDocumentCommand \calL { O{n} O{\alpha^{-n}\omega} }{
    \mathcal L^{(#1)}_{#2}
}
\NewDocumentCommand \Lab { O{a} O{b} }{
    \mathcal L_{#1\rightarrow#2}
}

\newcommand{\ubar}[1]{\text{\b{$#1$}}}

\title{Learning and music}
\author{George Lee}
\begin{document}
\maketitle
\section{Reasoning about tonal states}
The ear is generally assumed to pick up frequencies from 20 to 20000 Herz or so.
Here the training of a layer that maps pcm data to features that correspond to tonally useful data is discussed.
No claim of any kind of general structure of music is made, but some possible seemingly popular themes are used as inspiration.

Let $\Lambda\subset\mathbb R_{>0}$ represent the range of possible frequencies that we are interested in thinking about, with $a=\inf(\Lambda)<\sup(\Lambda)=b$.
Let $\Gamma=\{x\in\Lambda^{<\infty}:x_1<\cdots<x_{\texttt{len}(x)}\}$.
\subsection{A just approach}
It seems common in music to rely on simple ratios between different notes.
A just chord may be related to the possible ratios $\Phi=\{x\in\mathbb N^{<\infty}:x_1<\cdots<x_{\texttt{len}(x)},\text{hcf}(x)=1\}$.
Important just chords include:
\begin{itemize}
  \item an octave $1:2$.
    It is possible to restrict $\Lambda$ in this section to lie between unison and an octave, or even quotient out all intervals that are equivalent to multiples of $2$.
    Write $\sim_8$ for equivalent ratios under this equivalence
    Within the octave, every interval may be mapped to its ``inversion'' by multiplying the lower note by $2$ and switching their order.
  \item a ``perfect fifth/v'' $2:3$, which can be related to the ``perfect fourth'' by the inversion $2:3\sim_83:4$.
  \item 
\end{itemize}
\end{document}
