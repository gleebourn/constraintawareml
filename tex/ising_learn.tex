\documentclass[10pt,a4paper]{article}
\pdfoutput=1
\usepackage[margin=1in,footskip=0.25in]{geometry}

\usepackage{amsmath}
\usepackage{amssymb}
\usepackage{amsthm}
\usepackage{mathtools}
\usepackage{xparse}
\usepackage{cite}
\usepackage{hyperref}
\usepackage{cleveref}
\usepackage{geometry}
\usepackage{pgfplots}
\usepackage{listings}
\usepackage{tikz}
\usetikzlibrary{positioning}
%\usepackage[X2,T1]{fontenc}

\pgfplotsset{compat=1.16}

\newtheorem{claim}{Claim}
\newtheorem{rmk}{Remark}
\newtheorem{defn}[claim]{Definition}
\newtheorem{eg}[claim]{Example}
\newtheorem{lem}[claim]{Lemma}
\newtheorem{thm}[claim]{Theorem}
\newtheorem{cor}[claim]{Corollary}

\DeclareMathOperator{\spn}{span}
\DeclareMathOperator{\diag}{diag}
\DeclareMathOperator{\prj}{Proj}
\DeclareMathOperator{\mat}{Mat}
\DeclareMathOperator{\codim}{codim}

\DeclareMathOperator{\be}{\text{\fontencoding{X2}\selectfont б}}
\DeclareMathOperator{\ve}{\text{\fontencoding{X2}\selectfont в}}
\DeclareMathOperator{\ghe}{\text{\fontencoding{X2}\selectfont г}}

\newcommand{\xrightarrowdbl}[2][]{
  \xrightarrow[#1]{#2}\mathrel{\mkern-14mu}\rightarrow
}

\NewDocumentCommand \bern { O{k} }{
    \rho_{#1}
}

\NewDocumentCommand \calLp { O{n} O{\omega} }{
    {\mathcal L'}^{(#1)}_{#2}
}
\NewDocumentCommand \calL { O{n} O{\alpha^{-n}\omega} }{
    \mathcal L^{(#1)}_{#2}
}
\NewDocumentCommand \Lab { O{a} O{b} }{
    \mathcal L_{#1\rightarrow#2}
}

\newcommand{\ubar}[1]{\text{\b{$#1$}}}

\title{Learning Ising parameters for nonbinary classification}
\author{George Lee}
\begin{document}
\maketitle
\subsection*{Ising Likelihood maximisation}
Let $(X,y)\in\mathcal X^N\times\mathcal Y^N$, where $\mathcal X$ is some measure space and $\mathcal Y\subset_\text{finite}\mathbb R$.
Let $\sigma:\mathcal X\rightarrow\mathcal Y$.

For fixed $J:\mathcal X^2\rightarrow\mathbb R$ let

$$
H_J(X,y)=\sum_{i,j}J(X_i,X_j)y_iy_j.
$$
Given such an $H$ there is an associated probability distribution
$$
\widetilde{\mathbb P}_J(X,y)=e^{H_J(X,y)}~\text{and}~\mathbb P_J(X,y)=\frac{\widetilde{\mathbb P}(X,y)}{\sum_{\mathcal Y^N}\int_{\mathcal X^N}{\widetilde{\mathbb P}(X,y)}}=\frac{\widetilde{\mathbb P}(X,y)}{\mathcal Z_J},
$$
though the latter expression only makes sense where $\mathcal X$ is finite or with some infinitesimal interpretation.

This may be related to the usual Ising potential on a lattice $\Lambda$ by letting $\mathcal X=\Lambda$, letting $J(X,X')=1_{\{X~\text{is a neighbour of}~X'\}}$ and $\mathcal Y=\{\pm\}$.
Choosing some enumeration of $\Lambda$, a configuration
$$
\sigma:\Lambda\cong\{1,\cdots,|\Lambda|\}\rightarrow\{\pm\}
$$
has potential
$$
H(\sigma)=H_J\left((1,\cdots,|\Lambda|),\sigma\right).
$$

\bibliographystyle{plain}
\bibliography{imbalanced}
\end{document}
