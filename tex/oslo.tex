\documentclass{beamer}
\usepackage{amsmath}
\usepackage{algpseudocode}
\newcommand{\IN}{{\texttt{In}}}
\newcommand{\OUT}{{\texttt{Out}}}
\newcommand{\OUTSMOOTH}{{\texttt{OUT}_{\texttt{NN}}}}
\newcommand{\TWO}{\{\pm\}}
\newcommand{\PAR}{{\texttt{Par}}}

\newcommand{\GD}{{\texttt{GD}}}

\newcommand{\CO}{{P_{\texttt{cutoff}}}}

\newcommand{\UC}{{\texttt{C}_{\texttt{FPFN}}}}

\newcommand{\BS}{{P_{\texttt{bs}}}}
\newcommand{\RAND}{{\texttt{rand}}}
\newcommand{\SHUFF}{{\texttt{shuff}}}

\newcommand{\TRN}{{\texttt{Train}}}
\newcommand{\TST}{{\texttt{Test}}}

\newcommand{\RKG}{{\texttt{RKG}}}

\newcommand{\PRED}{{\texttt{Pred}}}
\newcommand{\PREDSMOOTH}{{\texttt{Pred}_{\texttt{NN}}}}
\newcommand{\yP}{{\hat y}}
\newcommand{\YP}{{\hat Y}}
\newcommand{\yN}{{\tilde y}}
\newcommand{\YN}{{\tilde Y}}
\newcommand{\UPD}{{\texttt{Upd}}}
\newcommand{\UPDSTEP}{{\texttt{Upd}_{\texttt{step}}}}
\newcommand{\UPDEP}{{\texttt{Upd}_{\texttt{epoch}}}}
\newcommand{\UPDFIT}{{\texttt{Upd}_{\texttt{fit}}}}

\newcommand{\XYEP}{{\texttt{Z}_{\texttt{epoch}}}}
\newcommand{\XB}{{\texttt{X}_{\texttt{B}}}}
\newcommand{\YB}{{\texttt{Y}_{\texttt{B}}}}

\newcommand{\FP}{{\texttt{FP}}}
\newcommand{\FN}{{\texttt{FN}}}
\newcommand{\FPB}{{\texttt{FP}_B}}
\newcommand{\FNB}{{\texttt{FN}_B}}
\newcommand{\FPT}{{\texttt{FP}_\oplus}}
\newcommand{\FNT}{{\texttt{FN}_\oplus}}

\newcommand{\CTHRESH}{{\texttt{C}}_{\texttt{FPFN}}}

\title{Constraint awareness for binary classification}
\subtitle{sample}
\begin{document}
\begin{frame}
\titlepage
\end{frame}

\section{Setup and motivation}
\begin{frame}
\frametitle{Setup}
\begin{itemize}
\item
  Objective: for a pair of random variables (RVs) $z=(x,y)$ taking values in $\IN\times\OUT$, seek to predict $y$ based on the value of $x$.
\item
  Focus on \textbf{binary classification}, where $\OUT=\TWO$ and $y$ is the label for $x$.
\item
  Depending on context and convenience, $\IN$ may be taken to be $\left(\texttt{float}\right)^{d_\IN}$ or $\mathbb R^{d_\IN}$.
\item
Focus on supervised classification, where we are provided with a labelled dataset of $N$ samples $Z=\left((X_i,Y_i)\right)_{i=1}^N\in\left(\IN\times\OUT\right)^N$
\item
Rows are assumed to be drawn iid, $(X_i,Y_i)\sim z$
\item
Write $p=\mathbb P(y=+)$ which can be estimated well provided $\tfrac1N\ll p$
\item
With that being said, focus on highly imbalanced datasets, ie cases where $p\ll1$
\end{itemize}
\end{frame}
\begin{frame}
\frametitle{Splitting}
\begin{itemize}
\item
In this setting an algorithm which trains models may be experimentally tested by first uniformly randomly splitting $Z$ by index into training and test datasets,
$$
\{1,\cdots,N\}=\TRN\sqcup\TST.
$$
\item
A model is then fit on $Z[\TRN]=\left((X_\TRN)_i,(Y_\TRN)_i\right)_{i=1}^{N_\TRN}$
\item
and validated on $Z[\TST]=\left((X_\TRN)_i,(Y_\TRN)_i\right)_{i=1}^{N_\TST}$.
\end{itemize}
\end{frame}
\begin{frame}
\frametitle{Learning algorithms}
\begin{itemize}
\item
  In our context a \textbf{learning method} (LM) with inputs in $\IN$ and outputs in $\OUT$ with parameter space $\PAR$ is a tuple $(\PRED,\UPD)$ consisting of \textbf{prediction} and \textbf{update} rules
\begin{gather*}
  \PRED:\PAR\times\IN\rightarrow\OUT\text{ and}\\
  \UPD:\PAR\times\left(\IN\times\OUT\right)^{<\infty}\rightarrow\PAR.
\end{gather*}
\item
We often make new algorithms from simpler ones by composition, summing or other routines.
\item
  When a dataset as above and parameters $P\in\PAR$ for an LM are fixed, denote the $i$th prediction $\YP_i=\PRED(P,X_i)$.
\end{itemize}
\end{frame}
\begin{frame}
\frametitle{Statistics associated with a parameterised model}
  \begin{itemize}
\item
  For fixed parameters $P\in\PAR$ the RV $\yP=\PRED(P,x)$ is the prediction made for $y$.
\item
In our setting where $\OUT=\TWO$, there are two possible types of errors.
Write
\begin{gather*}
  \FP=\mathbb P(\yP=+\text{ and }y=-)\text{ and }\\\FN=\mathbb P(\yP=-\text{ and }y=+).
\end{gather*}
for the \textbf{false positive} and \textbf{false negative} rates.
Omit $P$ where clear.
\end{itemize}
\end{frame}
\begin{frame}
\frametitle{User defined constraints}
\begin{itemize}
\item
  Aim: fit models that meet \textbf{user defined constraints} on the false positive and false negative rates.
\item
  Since $\FP$ and $\FN$ are probabilities, such a constraint can be written in the form
$$
    \UC=C(\FP,\FN)<\CTHRESH\text{ for some }C:[0,1]^2\rightarrow\mathbb R.
$$
\item
  Example: a model having error rates below fixed targets $\FPT,\FNT\in(0,1)$ is equivalent to requiring that $C(\FP,\FN)<0$ for the map
  $$
    C(s,t)=\texttt{max}(\tfrac s{\FPT},\tfrac t{\FNT})-1.
  $$
\item
  A real world user needs to specify constraints that depend upon both $\FP$ and $\FN$, unless a model that always predicts $+$ or $-$ is appropriate for their task
\end{itemize}
\end{frame}
\begin{frame}
\frametitle{From regressors to classifiers I}
\begin{itemize}
\item
Focus mainly on models based on neural networks (NNs)
\item
In this case a choice of parameters $P\in\PAR$ includes the network's weights $P_w\in\PAR_w$ and the parameters associated with some choice of gradient descent algorithm $P_\GD\in\PAR_\GD$
\item
  Problem: we want $\OUT=\TWO$ but the output in $\OUTSMOOTH$ of a NN is continuous/float valued
\item
  Solution: set the binary prediction equal to $+$ or $-$ depending on whether the NN's output $\PREDSMOOTH(P,X)$ is greater than a real parameter $\CO$: 
$$
\PRED(P,X)=\begin{cases}+\text{ if }\PREDSMOOTH(P,X)>\CO\text{ and}\\-\text{ otherwise.}\end{cases}
$$
\item
  Write $\yN=\PREDSMOOTH(P,x)$ and $\YN_i=\PREDSMOOTH(P,X_i)$.
\end{itemize}
\end{frame}
\begin{frame}
\frametitle{Gradient descent}
\begin{itemize}
\item
  For a NN with fixed forward pass implementation $\PREDSMOOTH$ and cutoff $\CO$ there are many possible choices of LMs which can be composed as we please to make new ones.
\item
  Fix a sufficiently well behaved loss function and gradient descent routine
    \begin{gather*}
      L:(\OUTSMOOTH\times\TWO)^{<\infty}\rightarrow\mathbb R\text{ and}\\
      \GD:\PAR_\GD\times\PAR_w\times\PAR_w\rightarrow\PAR_\GD\times\PAR_w.
    \end{gather*}
\item
A single update step $\UPDSTEP$ is given by
\begin{algorithmic}[0]
\Function{$\UPDSTEP$}{$P,\XB,\YB$}
  \State $(P_\GD,P_w)\gets\GD(P_\GD,P_w,\partial_w(L(\PRED(P,\XB),\YB)))$
  \State\Return $P$
\EndFunction.
\end{algorithmic}
\end{itemize}
\end{frame}
\begin{frame}
\frametitle{Epochs and fitting algorithms}
\begin{itemize}
\item
A fitting method $\UPDFIT$ typically repeatedly iterates the update step over the randomly batched training set, repeatedly calling an epoch learning method
\item
   For fixed random key generator $\RKG$ and batched shuffling routine $\SHUFF$
   % \begin{gather*}
   %   \RKG:\PAR_\RAND\rightarrow\PAR_\RAND\text{ and}\\
   %   \SHUFF:\mathbb N\times\PAR_\RAND\times\left(\IN\times\OUT\right)^{<\infty}\rightarrow\left(\IN\times\OUT\right)^{(<\infty)\times(<\infty)},
   % \end{gather*}
  such a routine is given by
\item
\begin{algorithmic}[0]
  \Function{$\UPDEP$}{$P$,$X$,$Y$}
  \State batched$\gets\SHUFF(\BS,P_\RAND,(X_i,Y_i)_i)$
  \ForAll{$(\XB,\YB)\in$ batched}
  \State $P\gets\UPDSTEP(P,\XB,\YB)$
  \EndFor
  \State $P_\RAND\gets\RKG(P_\RAND)$
  \State\Return $P$
  \EndFunction
\end{algorithmic}
\end{itemize}
\end{frame}
\begin{frame}
\begin{itemize}
\item
Larger problem: gradient descent relies on repeatedly nudging a model's weights down the slope of some smooth loss function.
We want to satisfy the user defined constraint, $C<0$.
While this quantity may be smooth, for a fixed batch $X_B,Y_B$ with $B=B_+\sqcup B_-$, the positive and negative labelled indixes rows these error rates can only be estimated, and the obvious choices
\begin{gather*}
\FPB=\frac{\#\left(\PRED(X_B)=+\text{ and }Y_B=-\right)}{\#B}\text{ and }\\
\FNB=\frac{\#\left(\PRED(X_B)=-\text{ and }Y_B=+\right)}{\#B}
\end{gather*}
are discrete quantities whose gradients are either $0$ or undefined.
\end{itemize}
\end{frame}
\begin{frame}
\frametitle{A base case: ``user defined constraint-unaware'' losses}
\begin{itemize}
\item
Models are commonly trained to minimise a loss which in some way represents how far predictions are from the target with no regard for the class, for example

\begin{gather*}
  L_{L^2}(B)=\sum_{i\in B} |\YP_i-Y_i|^2,\text{ or, if }\OUT_{\texttt{NN}}\subseteq(0,1),\\
  \texttt{BCE}(B)=-\tfrac1{\#B}\left(\sum_{i\in B_+}\log(\hat Y_i)+\sum_{i\in B_-}\log(1-\hat Y_i)\right)
\end{gather*}
\item
These choices of loss function place similar importance on making the model's predictions close to the actual value for every data point, and one may use them interchangably by rescaling the last layer of the NN appropriately.
\end{itemize}
\end{frame}
\begin{frame}
\frametitle{Constraint aware training I: class weighting}
\begin{itemize}
\item
  Relative rates of false positives and false negatives of the previous loss functions may be adjusted after training by a routine that chooses an appropriate $\CO$.
\item One may move between the loss functions describe above by rescaling the last layer of the NN appropriately.
  However, one may choose a loss $\CO$ which depends on target classes and instead make gradient descent steps that reduce the value of $C$ throughout the training process.
  Updates on the last layer of a network's bias have the effect of controlling the relative rates.
\item
  Seek to minimise an intuitive/explainable choice of loss function $L(B)=-\beta\sum_{i\in B_+}\hat Y_i+\tfrac1\beta\sum_{i\in B_-}\hat Y_i$.
  These losses are all increasing functions of the errors: neural networks are capable of learning nonlinear relationships in any case, and modern gradient descent methods also go some way toward mitigating the differences between different choices of loss.
\end{itemize}
\end{frame}
\begin{frame}
\frametitle{Experimental setup I}
\begin{itemize}
\item
Focus on malicious communication detection per UC2, where typically $p\ll 1$ and false positives may be far more tolerable and have less serious consequences than false negatives.
\item
A python class was developed to benchmark classification algorithms over a range of possible false positive and false negative targets
\item
$\texttt{ModelEvaluation}$ provides an interface for comparison of multiple classification algorithms
\item 
In order to benchmark binary classification performance over a range of problems, a dataset with greater than two classes may be provided: models are then trained to distinguish one class from many.

\item
$\texttt{ModelEvaluation}$ instantiates $\texttt{MultiTrainer}$s, each of which trains models with fixed hyperparameters but with varying end times, class labels and target error rates.
\end{itemize}
\end{frame}

\begin{frame}
\frametitle{Is resampling beneficial?}
\begin{itemize}
\item
Class weighting appears to achieve similar results while avoiding an extra step.
unsw-nb15:
\end{itemize}
\end{frame}
%\begin{frame}
%\end{itemize}
%\frametitle{...But are we learning likelihoods?}
%\begin{itemize}
%\item
%Consider the case where $g$ takes values in $[0,1]$, so that the threshold is to be chosen from $(0,1)$.
%\item
%A model is said to be \textbf{perfectly calibrated} if $\mathbb P(y=+|g(x)=t)=t$ for any $t\in[0,1]$.
%In this case, the output of the model can be understood as a probability that the label belongs to the positive class.
%\item
%This can be measured using the test set, for example by checking that over small intervals $[t-\delta,t+\delta]$ we have
%$$
%\frac{\#\{i\in\texttt{TEST}: |g(X_i)-t|<\delta\text{ and }Y_i=+\}}{\#\{i\in\texttt{TEST}: |g(X_i)-t|<\delta\}}\approx t.
%$$
%\end{itemize}
%\end{frame}
%\begin{frame}
%\frametitle{...But are we learning likelihoods? Continued}
%\begin{itemize}
%\item
%One should bear in mind that being well calibrated is no measure of the predictive abilities of a model!
%Note that if $g(x)=p$ for all values of $x$, then the model is perfectly calibrated but useless for classification.
%\item
%If $g$ is not perfectly calibrated, then it is often still possible to rescale its values $\tilde g(x)=h(g(x))$ such that $\tilde g$ is ``close'' to being perfectly calibrated.
%\end{itemize}
%\end{frame}
%\begin{frame}
%\frametitle{A criterion for the benefits of constraint aware binary classification}
%\begin{itemize}
%\item
%Idea: if we are able to find a good calibration $\tilde g$ of a model $g$, then the model has effectively learned likelihoods.
%\item
%In this case, we should be skeptical of the idea that we benefitted from a ``constraint aware'' training process, since we can still choose a threshold to meet any desired compromise between the false positive and false negative rates
%\end{itemize}
%\end{frame}
\end{document}
