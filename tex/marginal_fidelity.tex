\documentclass{article}
\usepackage{amsmath}
\usepackage{amsfonts}
\usepackage{algpseudocode}
%\newcommand{\IN}{{\texttt{In}}}
%\newcommand{\OUT}{{\texttt{Out}}}
\newcommand{\TYPE}[1][X]{{\texttt{#1}}}
\newcommand{\TYPEY}{{\TYPE[Y]}}
\newcommand{\IN}{\TYPE}
\newcommand{\OUT}{\TYPEY}
\newcommand{\DIM}{{\texttt{d}}}
\newcommand{\OUTSMOOTH}{{\texttt{OUT}_{\texttt{NN}}}}
\newcommand{\TWO}{\{\pm\}}
\newcommand{\PARS}{{\texttt{W}}}
\newcommand{\Par}{{w}}
\newcommand{\PAR}{{W}}
\newcommand{\BATCH}{{\texttt{B}}}
\newcommand{\dee}{\textrm{d}}

\newcommand{\GD}{{\texttt{GD}}}
\newcommand{\NN}{{\texttt{NN}}}
\newcommand{\siid}{\sim_{\texttt{iid}}}

\newcommand{\INDICES}{{\mathbb{I}}}
\newcommand{\Feature}{{x}}
\newcommand{\Target}{{y}}
\newcommand{\TARGET}{{Y}}
\newcommand{\FEATURE}{{X}}
\newcommand{\FeatureD}{{x'}}
\newcommand{\TargetD}{{y'}}

\NewDocumentCommand \FUNC {O{\Pred} O{x}}{{{#1}({#2})}}
\NewDocumentCommand \TS {O{\FEATURE} O{t}} {{#1}^{(#2)}}
\NewDocumentCommand \ind {O{n}}{{\texttt{#1}}}
\NewDocumentCommand \IND {O{\FEATURE} O{n}}{{#1}_{\ind[#2]}}
\NewDocumentCommand \REST {O{\FEATURE} O{\BATCH}}{{#1}_{#2}}
\NewDocumentCommand \ENUM {O{\FEATURE} O{\ind} O{\ROWS} O{\in}}{{\left({#1}_{#2}\right)_{{#2}{#4}{#3}}}}

\newcommand{\CO}{{\hat w}}
\newcommand{\NNW}{{\tilde w}}

\newcommand{\UC}{{\texttt{C}_{\texttt{FPFN}}}}

\newcommand{\BS}{{P_{\texttt{bs}}}}
\newcommand{\RAND}{{\texttt{rand}}}
\newcommand{\SHUFF}{{\texttt{shuff}}}
\newcommand{\DOM}{{\texttt{dom}}}
\newcommand{\Time}{{t}}
\newcommand{\TimeD}{{t'}}
\newcommand{\TIMES}{{\texttt T}}
\newcommand{\TIMEIND}[1][t]{{(#1)}}
\newcommand{\WEIGHTS}{{\texttt W}}
\newcommand{\ROWS}{{\texttt N}}

\newcommand{\TRN}{{\texttt{Train}}}
\newcommand{\TST}{{\texttt{Test}}}
\newcommand{\PRD}{{\texttt{Prod}}}

\newcommand{\RKG}{{\texttt{RKG}}}

\newcommand{\Phase}{{\omega}}
\newcommand{\PHASE}{{\Omega}}
\newcommand{\PhaseK}{{\widehat\Phase}}
\newcommand{\PHASEK}{{\widehat\PHASE}}
\newcommand{\STAT}{{S}}
\newcommand{\Stat}{{s}}
\newcommand{\StatD}{{\Stat'}}
\newcommand{\SCORE}{{C}}
\newcommand{\Score}{{c}}
\newcommand{\ScoreFun}{{\textbf{c}}}
\newcommand{\Regret}{{r}}
\newcommand{\Pred}[1][y]{{\hat{#1}}}
\newcommand{\PredD}[1][y]{{\hat{#1}'}}
\newcommand{\STATPRED}{\Pred[\STAT]}
\newcommand{\Predsmooth}[1][y]{{\tilde{#1}}}
\newcommand{\PRED}{{\Pred[\TARGET]}}
\newcommand{\PREDSMOOTH}{{\Predsmooth[\TARGET]}}
\NewDocumentCommand \Marginal{O{{}} O{\Target} O{\Feature} O{\TargetD}}{f_{\left({#2}_{#1}|{#3}_{#1}={#4}\right)}}
\NewDocumentCommand \Bestguess{O{{}} O{\Target} O{\FeatureD} O{\ScoreFun}}{{{#2}^{*}_{#1}\left({#3}_{#1},{#4}_{#1}\right)}}
\newcommand{\yP}{{\hat y}}
\newcommand{\YP}{{\hat \TARGET}}
\newcommand{\YN}{{\tilde \TARGET}}
\newcommand{\UPD}{{\texttt{Upd}}}
\newcommand{\BCE}{{\texttt{BCE}}}
\newcommand{\UPDSTEP}{{\texttt{Upd}_{\texttt{step}}}}
\newcommand{\UPDEP}{{\texttt{Upd}_{\texttt{epoch}}}}
\newcommand{\UPDFIT}{{\texttt{Upd}_{\texttt{fit}}}}
\newcommand{\Prob}{{\mathbb P}}
\newcommand{\ProbH}{{\widehat{\mathbb P}}}
\newcommand{\SGN}{{\texttt{sgn}}}

\newcommand{\XYEP}{{\texttt{Z}_{\texttt{epoch}}}}
\newcommand{\AB}{{A_\BATCH}}
\newcommand{\XB}{{\FEATURE_\BATCH}}
\newcommand{\YB}{{\TARGET_\BATCH}}
\newcommand{\ZB}{{Z_\BATCH}}
\newcommand{\YPB}{{\YP_\BATCH}}
\newcommand{\XTRN}{{\FEATURE_\TRN}}
\newcommand{\YTRN}{{\TARGET_\TRN}}
\newcommand{\YPTRN}{{\YP_\TRN}}

\newcommand{\fp}{{\texttt{fp}}}
\newcommand{\fn}{{\texttt{fn}}}
\newcommand{\FP}{{FP}}
\newcommand{\FN}{{FN}}
\newcommand{\FPB}{{\FP_\BATCH}}
\newcommand{\FNB}{{\FN_\BATCH}}
\newcommand{\CFP}{{\texttt{C}_\FP}}
\newcommand{\CFN}{{\texttt{C}_\FN}}
\newcommand{\FPE}{{\widehat{\texttt{FP}}}}
\newcommand{\FNE}{{\widehat{\texttt{FN}}}}
\newcommand{\Fp}{{\texttt{fp}}}
\newcommand{\Fn}{{\texttt{fn}}}
\newcommand{\FS}{\texttt f}
\newcommand{\FPT}{{\texttt{FP}_\oplus}}
\newcommand{\FNT}{{\texttt{FN}_\oplus}}

\newcommand{\CTHRESH}{{\texttt{C}}_{\texttt{FPFN}}}

\title{Loss-architecture duality and experimental design}
\author{George Lee}
\begin{document}

\maketitle


\subsection{Notational conventions}
\begin{itemize}
  \item
    \texttt{TEXTTT}/$\mathbb{MATHBB}$: Spaces
  \item
    $TEXT$: statistics for a fixed experiment, functions
  \item
    $text$: time indexes, numbers, RVs, (possibly) random functions
  \item
    Timeseries have bracketed superscripts eg $\TS[w]$
  \item
    \texttt{texttt}: discrete indices
  \item
    Discretely indexed statistics addressed with subscripts eg $\IND$
  \item
    $\hat h\hat a\hat t\hat s/\tilde t\tilde i\tilde l\tilde d\tilde e\tilde s$: predictions/continuous predictions
  \item
    $\INDICES=\{$all possible discrete indices$\}$, that is,
    $$
    \INDICES=\mathbb Z^{<\infty}=\bigsqcup_{n\in\mathbb N_0}\mathbb Z^n
    $$
  \item
    If $\BATCH\subseteq\ROWS$ and  $A\in\TYPE^\ROWS$ write $\REST[A]=(\IND[A][b])_{\texttt b\in\BATCH}\in\TYPE^\BATCH$ for the restriction of $A$ to $\BATCH$
\end{itemize}


\section{Problem definition and reformulation}
\subsection{Underlying dynamics}
\begin{itemize}
  \item
    User will be making inferences and decisions based on observations that are functions of the underlying state $\Phase$ of a system
  \item
    Correspondingly any statistic/data arising from the system is a function $\STAT(\Phase)$ of its state
  \item
    Observations and predictions made by user $\PhaseK=\ENUM[\PhaseK]$ depend on underlying system $\PhaseK=\PHASEK(\Phase)$
\end{itemize}


\subsection{Inferences from data}
\begin{itemize}
  \item
    Classification and regression: determine likely values of data $\TARGET$ taking discrete and continuous values respectively based on the value of some other data $\FEATURE$
  \item
    User works with dataset or stream $(\Feature,\Target)=(\ENUM[\Feature],\ENUM[\Target])$ indexed by $\ind\in\ROWS$
  \item
    All statistics available to user may be represented $\Feature=\ENUM[\FEATURE(\PhaseK_{\ind})][][\ROWS][\ind\in]$ and $\Target=\ENUM[\TARGET(\PhaseK_{\ind})][][\ROWS][\ind\in]$
\end{itemize}


\subsection{Formalising user needs}
\begin{itemize}
  \item
    Want to determine and represent constraints as user-explainable classifier optimisation problems
  \item
    Probablistic reasoning allows us to determine which continuous regression problems we should be solving
  \item
    Select appropriate model architectures and algorithms for these problems
  \item
    End user will evaluate performance of a model $\PRED$ with parameters $\Par$ by evaluating its predictions $\Pred=\PRED(\Par,\Feature)$ for feature $\Feature$ against the target $\Target$
  \item
    Suppose some statistic $\Score$ measuring performance of the classifier is made at time $t\in\TIMES$
  \item
    $\Score$ is a measure of a batch $\BATCH$ of earlier outcomes $\TargetD,\PredD=\REST[\Target],\REST[\Pred]$ which is chosen based on user observations and decisions $\PhaseK$
      $$
      \Score=\REST[\SCORE][\PhaseK](\TargetD,\PredD)
      $$
  \item
    If user knows the probabilities of possible outcomes conditional on $\Feature=\FeatureD$ then they should choose $\Pred(\Feature,\SCORE_{\PhaseK})$ that minimises $\SCORE_{\PhaseK}$ in expectation
  \item
    For binary classification, this is the choice $\Pred\in\TWO^{\BATCH}$ that minimises
      $$
      \mathbb E(\SCORE_{\PhaseK}(\Target,\Pred)|\Feature=\FeatureD)=\sum_{\TargetD\in\TWO^{\BATCH}}\SCORE_{\PhaseK}(\TargetD,\Pred)\mathbb P(\Target=\TargetD|\Feature=\FeatureD)
      $$
  \end{itemize}
  

\subsection{Theoretical characterisation of optimal predictions: marginals}
  \begin{itemize}
  \item
    Have reformulated classification probablistically, allowing us to write down an optimal solution in terms of a probability
    \item
      Let $(\Feature,\Target)$ be RVs taking values in $\IN$ and $\OUT$
    \item
      Marginal distribution represents best knowledge of what $\Target$ is likely to be if we only know the value of $\Feature$
    \item
        Write $\Marginal$ for the density/probability that $\Target=\TargetD$ conditional on the observation $\Feature=\FeatureD$
  \item
    If we know the marginals of $\Target$ given $\Feature$ then we can also calculate the marginals for any statistic $\Stat=\STAT(\Target)$ of $\Target$
      %with the rule
      %$$
      %\Marginal[][\Stat][\Feature][\FeatureD](\StatD)=\int_{\TargetD\in\OUT:\STAT(\TargetD)=\StatD}\Marginal[][\Target][\Feature][\FeatureD](\TargetD)\frac{\dee\STAT(\TargetD)}{\dee\TargetD}\dee\TargetD
      %$$
  \end{itemize}


\subsection{Perfect knowledge of marginals allows for optimal decision making}
  \begin{itemize}
        \item
      The score for a single binary classification $\SCORE_\PhaseK$ is defined by four numbers for the four possible prediction-target outcomes, can rescale to no costs for correct choice without changing the problem, so just interested in two error rates
        \item
          the marginal distribution is equivalent to its corresponding marginal probability $\mathbb P(\Target=+|\Feature=\TargetD)$ of a positive classification
    \item
      At each step optimal prediction $\Pred\in\TWO$ is that with a smaller value of
          $$
          \mathbb E(\SCORE_\PhaseK(\Target,\pm)|\Feature=\FeatureD)=\SCORE_\PhaseK(\mp,\pm)\mathbb P(\Target=+|\Feature=\FeatureD)
          $$
      \end{itemize}


\subsection{Optimal regression targets for offline learning}
  \begin{itemize}
  \item
    If all we know is that mistakes will be bad,
      \begin{itemize}
        \item
          impossible to refine problem beyond the fact that good knowledge of the marginals will allow probablistic optimisation of stats on the fly
        \item
          User may attempt to learn the marginals - equivalent to minimising AUC(ROC)
      \end{itemize}
  \end{itemize}
  
  
    \subsection{Preemptive learning when $C$ can be forecast}
  \begin{itemize}
  \item
    If it is at least possible to predict likely values for relative cost of $\FP$ versus $\FN$ may be possible to restrict attention to some function of the ROC
  \item
    Sufficient to learn marginals to high precision only over the range relevant to likely cost ratios $\frac{\SCORE_\PhaseK(\pm,\mp)}{\SCORE_\PhaseK(\mp,\pm)}$ and imbalances $\mathbb P(+)$
  \item
    Any other choice of target will be distributed according to some function of the marginals
  \item
    
  \end{itemize}
  \subsection{Experiment I}
  \begin{itemize}
    \item
      Want to compare training algorithms for fixed ReLU network
    \item
      Models scored on cost in cases where cost of false negatives $\gg$ that of false positives
    \item
      Three approaches
      \begin{itemize}
        \item
          Resampling and BCE
        \item
          No resampling and BCE
        \item
          No resampling and cost-derived loss
      \end{itemize}
  \end{itemize}

%
%\subsection{Offline and online decision making}
%  \begin{itemize}
%    \item
%      Offline: start with train and test indices $\TRN\sqcup\TST$. user will
%      \begin{itemize}
%        \item
%          train a model on $\TRN$
%        \item
%          evaluate performance on $\TST$
%        \item
%          decide based on this whether to use the model to make prediction on real world rows whose labels have not yet been identified $\PRD$.
%
%    \end{itemize}
%  \end{itemize}
%
%
%\subsection{Setup I}
%  \begin{itemize}
%  \item
%    $\STAT$ should be as simple and explainable to the user as possible
%  \item
%  To make the problem well defined, need to properly specify $\FPE$ and $\FNE$
%    \item
%      In application: model used to predict labels $\TARGET_i$ of a steam of real world observation $\FEATURE_i$.
%    \item
%      User could take many different approaches, assume based on a finite batch $\BATCH$ of predictions $\STATPRED_B(\FPE_B,\FNE_B)$
%    \item
%      We can always assume that errors are bad, so that $\STATPRED_B$ is increasing in the values $\FPE_B$ and $\FNE_B$.
%    \item
%      Offline setting: will be fed rows and labels $(\FEATURE_i,\TARGET_i)$ sampled according to
%  \end{itemize}
%  
%  
%  \begin{itemize}
%\item
%  Approach: build a classifier by turning the problem into one of
%  \begin{itemize}
%\item
%  finding a regression model $\tilde y$ trained to minimise a related tractable quantity $L$ then obtaining a binary prediction $\hat y$ depending on the value of $\tilde y$.
%  \end{itemize}
%\item
%  NN with forward pass $\Predsmooth:\IN\times\PAR\rightarrow\OUT\subseteq\mathbb R$ applied to feature $x$ to predict label $y$.
%\item
%  For such a fixed map, when we speak of training $\tilde y$ we mean iteratively applying $\GD$ for fixed architecture $\tilde y\in\NN(\PAR,\IN,\OUT)$ to find weights $w\in\PAR$ to solve some optimisation problem
%\item
%  Fix a batch of rows $(x,y)$
%\item
%  For a batch $\BATCH$ of one or more rows the NN's output can be encoded as $(\Fp_\BATCH,\Fn_\BATCH)$ where for each row one of $\Fp$ and $\Fn=0$ while the other increases with how far away from $y$ the prediction is for $y=+$ and $-$ respectively.
%  For a single row $i$ of a dataset write $\FS_i$ for whichever of $\Fp_i$ and $\Fn_i$ is nonzero
%\item
%  Depending on the ranges of $\Predsmooth(x)$ and $y$ we may define various losses
%\end{itemize}
%
%
%\subsection{Setup III}
%  \begin{itemize}
%\item
%  Define $\SGN(x)=\begin{cases}+\text{ if }x\geq0,\\-\text{ otherwise}\end{cases}$
%\item
%  $L^p$ loss $L_a=\sum_i\FS_i^p$ for $1\leq p\leq\infty$
%\item
%  dot weighted loss $L_b=\Predsmooth\cdot y$
%\item
%  weighted cost $L_c=\sum_i{\beta_c}^{y_i}\FS_i$ for $\beta>0$
%\item
%  weighted binary cross entropy $L_d=-\sum_i{\beta_d}^{\SGN(y_i)}\log(1-\FS_i)$ for $\beta>0$
%\item
%  What difference do they make?
%\end{itemize}
%
%
%\subsection{Transforming out different choices of losses}
%  \begin{itemize}
%    \item
%      When designing an algorithm based on an NN we might wish to focus on a choice of loss function with no consideration for the specific choice of NN architecture
%    \item
%      This isn't possible.
%    \item
%      Each choice of loss function defined above is a sufficiently nice, increasing function of each $\Fp_i$ and $\Fn_i$.
%    \item
%      This means that for any two of these functions $L_u$ and $L_v$ there is a transformation $\varphi_{uv}:\DOM(L_u)\rightarrow\DOM(L_v)$ such that $L_v(\varphi_{uv}(\Predsmooth))=L_u(\Predsmooth)$
%\end{itemize}
%
%
%\subsection{Loss functions can be viewed as equivalent if we don't fix the NN architecture!}
%  \begin{itemize}
%    \item
%      Write $\Predsmooth_{uv}=\varphi_{uv}(\Predsmooth)$.
%    \item
%      Changing the loss function is equivalent to rescaling the last layer!
%    \item
%      There is an equivalence between
%      \begin{itemize}
%      \item
%        modifying the last layer activation (NN architecture perspective), and
%      \item
%        changing the choice of loss (gradient descent tuning perspective).
%      \end{itemize}
%    \item
%      When defining an algorithm to train an architecture, we need to restrict the class of models we are using - otherwise there is no way to say which loss is better, because we can always find an architecture that performs arbitrarily well or poorly for a fixed loss by rescaling.
%    \item
%      I is impossible to argue rigorously in favour of choosing any of the above loss functions without also understanding the exact workings of $\tilde y$.
%    \item
%      This means that training $\Predsmooth_{uv}$ with the loss function $L_u$ is equivalent to training $\Predsmooth$ with loss function $L_v$ then using $\Predsmooth_{uv}$ for the forward pass!
%    \item
%      In this context, changing loss functions is equivalent to changing the last layer's activation.
%    \item
%      A good NN architecture is likely able to fit to the underlying transformations $\varphi_{uv}$ as part of the learning process anyway
%    \item
%      If we do need to investigate custom loss functions, need to also consider model architecture
%%      Let $\Predsmooth_{uv}(x)=\varphi_{uv}(\Predsmooth(x))$ denote the output of a $NN$ defined by applying $\Predsmooth$ then applying $\varphi$ to the result.
%%    \item
%%      It then holds that $L_v(\Predsmooth_{uv})=L_u(\Predsmooth)$.
%%
%  \end{itemize}
\end{document}

%\bibliography{imbalanced}{}
%\bibliographystyle{plaine
